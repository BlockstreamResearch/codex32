\documentclass{article}
\usepackage[english]{babel}
\usepackage{enumerate}

%%%%%%%%%% Start TeXmacs macros
\newcommand{\tmem}[1]{{\em #1\/}}
\newcommand{\tmsamp}[1]{\textsf{#1}}
\newcommand{\tmstrong}[1]{\textbf{#1}}
\newcommand{\tmverbatim}[1]{{\ttfamily{#1}}}
\newenvironment{enumeratenumeric}{\begin{enumerate}[1.] }{\end{enumerate}}
%%%%%%%%%% End TeXmacs macros

\begin{document}

\part{High-Level Introduction?}\part{Detailed Instructions}

\section{Tables or Volvelles}

Hand computation for the procedures in this document can be performed either
by using the Principal Tables to look up values, or by using volvelle wheels
to look up values. \ While the volvelle wheels take time to cut out and
assemble, they are generally easier to use than the tables when available.

\section{Share Format}

For 128-bit secret seeds, each share is 48 characters long. Shares begin with
the three character prefix \tmverbatim{MS1}. This is followed by a six
character header. The next 26-characters is the data portion. The last
13-characters is the checksum.

The header consists of:
\begin{itemize}
  \item The {\tmstrong{threshold}} which is the value $k$, a digit between
  \tmverbatim{2} and \tmverbatim{9} inclusive, however Module 0 only supports
  $k \leq 3$. When secret splitting is not used, the a \tmverbatim{0} digit is
  placed here instead.
  
  \item The {\tmstrong{identifier}} which is four bech32 characters.
  
  \item The {\tmstrong{share index}} which is any bech32 character except for
  \tmverbatim{S}. The \tmverbatim{S} index is the {\tmstrong{secret index}}.
  The data portion of the {\tmstrong{secret index}} contains the secret seed.
\end{itemize}
Shares of one secret all have the same threshold and identifiers. If you have
multiple secrets, you should use distinct identifiers for each secret so as
not to mix-up shares of different secrets with each other. The identifiers are
not considered secret themselves.

\section{New Secret Seed}

Generating a $k$-of-$n$ scheme for a new random secret is most easily done by
generating fresh random shares directly. This process generates a new random
secret seed without directly revealing it.
\begin{enumeratenumeric}
  \item Choose a threshold $k$ and total number of shares $n$ that suits your
  needs. The threshold $k$ must be 3 or less and $n$ must be 31 or less.
  
  \item Choose a 4 character identifier for your new secret seed. The
  identifier can be anything as long as it only uses the Bech32 character set.
  The identifier itself is not secret. However, the identifier should be
  unique for each secret seed.
  
  \item Follow Section~\ref{NewSecret1} to generate the first $k$ shares.
  
  \item Follow Section~\ref{NewSecret2} to generate the remaining $n - k$
  shares.
  
  \item Load your shares into your BIP-???? compliant wallet or use the
  Recover Secret Seed procedure in Section~\ref{RecoverSecret} to access your
  new secret seed value.
  
  \item Copy and distribute your $n$ shares into safe and secure locations.
  Remember that you will need to recover at least $k$ of these shares to
  recover your secret seed. Also remember that anyone else who recovers $k$ of
  these shares can also recover your secret seed.
  
  \item Securely dispose of all worksheets you used in the generation
  procedure. If these worksheets are not securely disposed of, the could be
  used to recover your secret seed.
\end{enumeratenumeric}
\subsection{New Secret Seed: Stage 1}\label{NewSecret1}

Make $2 k$ copies of the Checksum Worksheet and save half of them for later.

Fill out the header portion of $k$ many Checksum Worksheets with your chosen
threshold $k$ and chosen ID. Place a unique share index on each worksheet
starting with share \tmverbatim{A} on the first worksheet, \tmverbatim{C} on
the second worksheet, and so on through the $k$ first characters from the
Bech32 character set. (Note that \tmverbatim{B} and \tmverbatim{I} are not
part of the Bech32 character set and are omitted). However, if you are not
splitting your secret, (i.e. $k = 1$) see the special instructions below.

Fill out the 26 character data portion of each Checksum Worksheet with random
characters. Use the Random Character Worksheet to generate each random
character.

Follow the Checksum Worksheet instructions to generate a checksum for each
worksheet.

{\tmstrong{Critical Step:}} Verify your checksum by copying each the 48
character share onto an empty worksheet that you saved earlier. Follow the
checksum verification instructions to verify each checksum. If any checksum
fails to verify then make more copies of the Checksum worksheet and redo the
checksum generation and checksum verification steps again.

{\tmstrong{Failure to verify each checksum may lead to irrecoverable loss of
the secret seed and funds.}}

{\tmem{Special rules for k=1}}: If you are not splitting your secret, then use
a \tmverbatim{0} digit in the threshold place, and use the \tmverbatim{S}
character in the share index place. Follow the same instructions for
generating the data portion and the checksum.

\subsection{New Secret Seed: Stage 2}\label{NewSecret2}

The remaining $n - k$ are derived from the first $k$ shares using the addition
worksheet corresponding to the $k$ value you have chosen. Label the entries of
the addition worksheet with the share indices that you will be using. We
recommend following the Bech32 character order following the last index you
generated in Stage 1.

Use the following procedure to derive a new share:
\begin{enumeratenumeric}
  \item Make a copy of the Addition Worksheet for the value of $k$ that you
  are using and label the shares with the share indices from the shares you
  have already generated, \tmverbatim{A}, \tmverbatim{C} and \tmverbatim{D} if
  $k = 3$. Label the Final Share Index with the new share index you want to
  derive.
  
  \item In the derivation table for your value of $k$, find the column
  corresponding to the new share index you want to derive. Fill in the symbols
  on the Addition Worksheet with the symbols from that column next to the
  share index for each row.
  
  \item Follow the Translation Worksheet instructions derive the new share.
\end{enumeratenumeric}
\section{Recover Secret Seed}\label{RecoverSecret}

Normally you would not recover a secret seed yourself, and instead load shares
into a BIP-???? compliant wallet. However, you can recover the secret seed by
hand if no compatible wallets are available or whatever other reason you might
have.

The recovery procedure uses exactly $k$ many shares. If you have more than $k$
many shares, you can select any $k$ of them and set the other shares aside.

Use the following procedure to recover the share:
\begin{enumeratenumeric}
  \item For each share, fill in a Checksum Worksheet and verify the checksum.
  If a checksum fails to verify, you may have made an error on your worksheet,
  or there may be an error in your share data. If there is an error in your
  share data, you can try substituting the share with a different one.
  Otherwise you will need to perform the Error Correction Procedure on your
  share, which will involve the assistance of a digital computer.
  
  \item Make a copy of the Addition Worksheet for the value of $k$ that you
  are using and label the shares with the share indices from the shares you
  have selected to recover from, and label the Final Share Index as
  {\tmsamp{S}}.
  
  \item You can fill in the symbols for each share on the Addition Worksheet
  using either the table lookup, or the volvelle lookup:
  
  {\tmem{Table lookup $k = 2$}}: Fill in the symbol from the Recover table by
  finding the column with the associated share, and the row for the other
  share.
  
  {\tmem{Volvelle lookup $k = 2$}}: Turn the Recovery Volvelle to point to the
  share being considered. Find the symbol pointed to under the other share
  index on the wheel and fill in that symbol next to the share we are
  considering on the Addition Worksheet.
  
  {\tmem{Table lookup $k = 3$}}: Finding the column with the associated share.
  Lookup the two symbols from the two rows corresponding to the two other
  shares. Make a note of these two symbols on a scrap piece of paper. Use the
  multiplication table to multiply the the two symbols and fill in that
  share's symbol on the Addition Worksheet with the resulting product.
  
  {\tmem{Volvelle lookup $k = 3$}}: Turn the Recovery Volvelle to point to the
  share being considered. Find the two symbol pointed to under the other share
  indices on the wheel. Turn the multiplication wheel to the first of these
  two symbols. Find the second symbol on the lower ring, and lookup the symbol
  it is pointing to. Fill that symbol next to the share we are considering on
  the Addition Worksheet.
  
  \item Repeat step 3 for each share on the Addition worksheet.
  
  {\tmem{Tip}}: For $k = 2$ the two symbols will always be opposite each other
  on the Recovery Volvelle and are connect by a grey line.
  
  \item Follow the Translation Worksheet instructions recover the secret
  share.
  
  \item After completing the checksum verification you may run the Binary
  Worksheet on the secret share to convert the secret seed into binary format.
\end{enumeratenumeric}

\section{Random Character Worksheet}

This procedure generates random Bech32 characters from dice using a debiasing
technique. As long as the procedure is followed carefully and correctly, even
low-quality consumer dice can be safely used to generate 128-bits of near
perfect randomness.

You will need the following items:
\begin{itemize}
  \item The character tree below.
  
  \item The dice track page.
  
  \item Five distinct and distinguishable dice.
  
  \item Six coins or other small markers.
  
  \item One cup for shaking and rolling the dice.
\end{itemize}
It is important that the five dice be distinguishable. They can be different
colours or, if you have adventure dice, you they can have different number of
sides. Label the five dice tracks by the colour or other distinguishing
features of each die.

Use the following procedure to generate one random character:
\begin{enumeratenumeric}
  \item Place one coin at the top of the character tree.
  
  \item Place the five dice together in the cup, shake, and roll the dice.
  Mark the results of each die by placing a coin on the track corresponding to
  each distinguished die's result.
  
  \item Place the five dice together in the cup again, shake, and roll the
  dice. Mark the results of each die by placing the die itself on its own
  track at the value it rolled.
  
  \item If any die and maker on on the same spot on its track, remove
  {\tmstrong{both}} the marker and the die. Repeat steps 2 and 3 for any
  removed dice to replace the marker and the die. Keep repeating until it is
  no die and marker on on the same spot on the track.
  
  \item Starting with the first track, move the coin on the character tree
  down and to the left, if the die on the first track is to the left of the
  marker. Otherwise, if the die on the first track is to the right of the
  marker, move the coin on the character tree down and to the right.
  
  \item Repeat step 5 for the second, third, forth, and fifth dice tracks.
  
  \item Record the character under where the coin on the character tree ends
  up on as your random character.
  
  \item Clear of all dice and markers. Return to Step 1 if you need to
  generate more characters.
\end{enumeratenumeric}
{\tmstrong{It is critical in step 4 to remove both the dice and marker.}}

\section{Checksum Worksheet}

\section{Translation Worksheet}

The Translation Worksheet is used by both the secret recovery procedure and
the share derivation procedure. This procedure translate $k$ shares and sum
the results to get produce a share. Before starting the Translation Procedure
the Translation Worksheet should already have $k$ many share index and symbols
filled in.

The Translation procedure can either be done using lookup tables or using
volvelles.
\begin{enumeratenumeric}
  \item Translate the first share using the first symbol.
  
  {\tmem{Lookup Table Method}}: Find the column in the Translation table for
  the symbol. Translate each character of the share after the \tmverbatim{MS1}
  prefix by looking up the row for that character and writing the resulting
  character.
  
  {\tmem{Volvelle Method}}: Turn the multiplication wheel to the symbol, then
  flip the disc over to the translations side. Translate each character of the
  share after the \tmverbatim{MS1} prefix by looking up the character on the
  lower ring and writing the resulting character it points to.
  
  \item Repeat Step 1 for the second share using the second symbol.
  
  \item Add the two translated shares, character by character.
  
  {\tmem{Lookup Table Method}}: For each position, find the row/column of two
  translated characters in the Addition Table and write the resulting
  character in the position below them. Addition is symmetric so the two
  characters can be looked up in either order.
  
  {\tmem{Volvelle Method}}: For each position, turn the addition wheel to the
  character of one translation and lookup the character from the other
  translation in the interior of the wheel. Write the resulting character in
  the position below them. Addition is symmetric so two characters can be
  looked up in either order.
  
  \item If $k > 2$ then repeat steps 1-3, translating each additional share
  and adding it to the previous sum until all shares are translated and added
  together.
  
  \item {\tmstrong{Critical Step:}} Verify the resulting share. It should have
  a correct header with a correct threshold, identity, and have the correct
  share index. Copy the share into a fresh checksum worksheet and follow the
  checksum verification step. If the checksum is not valid then you have made
  and error either in the checksum verification procedure or in the share
  derivation procedure and you must repeat the procedure.
\end{enumeratenumeric}
{\tmem{Tip:}} You are more likely to make an error in the checksum
verification procedure itself. To help guide the checksum verification you can
run the translate procedure on the bottom diagonal of the checksum worksheets
for each share. Fill in the resulting in the bottom diagonal of the
verification worksheet of your new share. As you proceed with the checksum
verification procedure you should encounter the same character on the lower
diagonal that you have prefilled. If not, you have made an error, either in
the checksum verification up to that point, or in a column of your addition
worksheet that is before that point.

\section{Binary Conversion Worksheet}

\section{Error Correction}

\section{Volvelle Assembly}

\end{document}
